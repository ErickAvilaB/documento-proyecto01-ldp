% Configuración del documento

% ---------- Paquetes necesarios ----------
\usepackage[utf8]{inputenc} % Soporte para caracteres
\usepackage[spanish,mexico]{babel} % Soporte en español
\usepackage[letterpaper, margin=2cm]{geometry} % Soporte para tamaño de hoja y margenes
\usepackage{graphicx} % Soporte para imágenes
\usepackage{amsmath} % Soporte para matemáticas
\usepackage{amssymb} % Soporte para símbolos
\usepackage{amsthm} % Soporte para teoremas
\usepackage{setspace} % Soporte para entrelineado
\usepackage{caption} % Soporte pie de imagen
\usepackage{fancyhdr} % Soporte para encabezados y pie de pagina

% ---------- Configuraciones ----------
\graphicspath{ {img/} } % Carpeta de imágenes
\onehalfspacing % Entrelineado 1.5

% Estilo de la pagina
\fancypagestyle{miEstilo}{% ====== Paquetes extra necesarios para la sección de Sintaxis léxica ======
\usepackage[T1]{fontenc}     % Salida en T1 (guiones, tt, etc.)
\usepackage{lmodern}         % Fuente compatible con T1
\usepackage{microtype}       % Mejora de espaciado
\usepackage{enumitem}        % Listas compactas (usado en la sección)
\usepackage{booktabs}        % Tablas bonitas (tabla de palabras reservadas)
\usepackage{array}           % Columnas personalizadas en la tabla
\usepackage{fancyvrb}        % Entorno verbatim 'code' para regex y ejemplos
\usepackage{hyperref}        % Enlaces y \texorpdfstring en títulos
\hypersetup{colorlinks=true,linkcolor=black,urlcolor=blue}

% Entorno verbatim 'code' que usa la sección
\DefineVerbatimEnvironment{code}{Verbatim}{fontsize=\small,commandchars=\\\{\}}
\fancyhf{} % Resetea estilos
\fancyhead[L]{\autorAPA} % Nombre de la tarea
\fancyhead[R]{\tareaName} % Nombre de la tarea
\renewcommand{\headrulewidth}{0.01cm} % Linea arriba
\renewcommand{\footrulewidth}{0.01cm} % Linea abajo
\fancyfoot[C]{\thepage} % Número de página
}

\fancypagestyle{portada}{
    \fancyhf{} % Resetea estilos
    \renewcommand{\headrulewidth}{0.01cm} % Linea arriba
    \renewcommand{\footrulewidth}{0.01cm} % Linea abajo
    \fancyfoot[C]{\thepage} % Número de página
}

\pagestyle{miEstilo} % Establece el estilo de la pagina

% ====== Paquetes extra necesarios para la sección de Sintaxis léxica ======
\usepackage[T1]{fontenc}     % Salida en T1 (guiones, tt, etc.)
\usepackage{lmodern}         % Fuente compatible con T1
\usepackage{microtype}       % Mejora de espaciado
\usepackage{enumitem}        % Listas compactas (usado en la sección)
\usepackage{booktabs}        % Tablas bonitas (tabla de palabras reservadas)
\usepackage{array}           % Columnas personalizadas en la tabla
\usepackage{fancyvrb}        % Entorno verbatim 'code' para regex y ejemplos
\usepackage{hyperref}        % Enlaces y \texorpdfstring en títulos
\hypersetup{colorlinks=true,linkcolor=black,urlcolor=blue}

% Entorno verbatim 'code' que usa la sección
\DefineVerbatimEnvironment{code}{Verbatim}{fontsize=\small}